\documentclass[a4paper,10pt]{article}
\usepackage{polski}
\usepackage[utf8]{inputenc}

\title{Obliczanie wartości pochodnej funkcji na podstawie
jej wartości w skończonej liczbie punktów}
\author{Aleksander Czeszejko-Sochacki}

\begin{document}

 \maketitle
 \pagenumbering{arabic}
 \section{Wstęp}
  Problem wyliczania pochodnej funkcji na podstawie jej wartości w
  skończonej liczbie punktów jest dość popularny, szczególnie
  w zagadnieniach fizycznych. Powiedzmy, że mamy bardzo wiele obserwacji
  i przypuszczamy, że układają się one w pewną zależność. Chcemy obliczyć,
  jak ta zależność się zmienia (np. w czasie). Jest to analogiczny problem
  do tego, który będę rozważał w tym sprawozdaniu.
 
 \section{Cel wykonanych doświadczeń}
  Wpierw chcemy ustalić, czy dla każdej funkcji da się przybliżyć pochodną
  w danym punkcie na podstawie wartośći w węzłach. Sprawdzimy, jak dokładność przybliżenia zależy od 
  rozmieszczenia i liczby węzłów. Ocenimy poprawność i dokładność testowanych
  metod.
  
  \section{Opis użytych metod}
   Do obliczania przybliżonej wartości pochodnej użyłem dwóch metod (zwanych
   w dalszej części odpowiednio metodą 1. i metodą 2.):
   
   \begin{itemize}
    \item Przybliżania pochodnej w punkcie jako ilorazu różnicowego na przedziale, 
    do którego ten punkt należy
    \item Obliczanie pochodnej wielomianu interpolacyjnego Newtona w punkcie
    przechodzącego przez dane węzły
   \end{itemize}
   
   Zakładamy dalej, że $X = [x_1, x_2, \dots, x_n]$ - zbiór wezłów, przy
   czym $x_1 < x_2 < \dots < x_n$ oraz a -punkt, w którym chcemy
   znać przybliżenie pochodnej.
   
   \subsection{Metoda obliczania ilorazu różnicowego na przedziale}
   \begin{center}
    \textbf{Schemat metody}
   \end{center}
   
   \begin{enumerate}
    \item Znajdź $i \in \{1, 2, \dots, n\}$ takie, że $a \in <x_i, x_{i+1}>$
    \item f'(a) = $\frac{f(x_{i+1}) - f(x_i)}{x_{i+1} - x_i}$
   \end{enumerate}
   
   \indent Krok 1. został zrealizowany w pliku \texttt{program.ipynb} funkcją
   \texttt{search\_bounds(parametry)}.
   \par
   \indent Krok 2. został zrealizowany w pliku \texttt{program.ipynb} funkcją
   \texttt{approx\_derivate(parametry)}.
   \par
   
   \subsection{Metoda obliczania pochodnej wielomianu interpolacyjnego
   Newtona}
   \begin{center}
    \textbf{Schemat metody}
   \end{center}
   
   \begin{enumerate}
    \item Wylicz tablicę B kolejnych ilorazów różnicowych:
    \[B = [f[x_1], f[x_1, x_2], f[x_1, x_2, x_3], \dots, f[x_1, x_2, \dots, x_n]]\]
    \item Oblicz na podstawie X i ilorazów różnicowych wielomian interpolacyjny
    Newtona w postaci potęgowej:
    \begin{enumerate}
     \item Niech newton(x) = B[0]
     \item Oblicz kolejny wielomian węzłowy p(x) w postaci potęgowej
     na podstawie poprzedniego (pierwszy jest tożsamościowo równy 1)
     \item Dodaj do wielomianu newton(x) wielomian p(x) przemnożony
     przez kolejny iloraz różnicowy z tablicy B
     \item Powtarzaj kroki b) - c) dopóki nie przeiterujesz całej tablicy B
     \item Zwróć wynik: newton(x)
    \end{enumerate}
    
    \item Oblicz pochodną wielomianu newton(x), f'(a) = newton'(a)
   \end{enumerate}
   
   Krok 1. został zrealizowany w pliku \texttt{program.ipynb} funkcją
   \texttt{coefs(parametry)}
   
   Krok 2. został zrealizowany w pliku \texttt{program.ipynb} funkcją
   \texttt{newton\_interpolation(parametry)}
   
   Krok 3. został zrealizowany w pliku \texttt{program.ipynb} funkcją
   \texttt{newton\_derivate(parametry)}\\
      
   Przypuszczamy, że druga metoda będzie dawała rezultaty bliższe dokładnej 
   wartości pochodnej niż pierwsza.
   
   \section{Testy}
    W pliku \texttt{program\_ipynb} wykonałem szereg testów powyższych metod
    dla funkcji: \texttt{sin(x), ln(x), exp(x)} oraz \texttt{2(x + 1)(x - 3)}.
    Obliczenia przeprowadziłem w arytmetyce BigFloat w języku julia, aby uniknąć
    niedokładności wyników, które mogłyby być spowodowane zbyt dużą precyzją
    arytmetyki. Testy zostały przeprowadzone dla różnych zbiorów węzłów w przedziale
    [0, 10] (dla funkcji \texttt{exp(x)} również w punkcie spoza niego: 30).
    
    \subsection{\texttt{sin(x)}}
     Po pierwsze \texttt{sin(x)} w przedziale [0, 10] jest funkcją gładką, przyjmującą skończone,
     na dodatek niewielkie wartości (co do modułu). To znaczy, że nie powinno być problemu
     z dokładną interpolacją tej funkcji na tym przedziale. Testy to potwierdzają.
     \begin{enumerate}
      \item W kazdym teście wraz ze zwiększeniem liczby węzłów
      odnotowujemy zmniejszenie błędu względnego (przybliżonej wartości do wartości
      dokładnej). 
      \item Metoda 2. dawała lepsze wyniki niemal we wszystkich przypadkach
      (wyjątek - 11 węzłów)
      \item Co więcej, w każdym teście dla 101 węzłów błąd względny wynosi 0.0 (w precyzji
      arytmetyki BigFloat)
     \end{enumerate}
     
    \subsection{\texttt{exp(x)}}
     Funkcja \texttt{exp(x)} jest funkcją gładką. Jednak bardzo szybko rośnie. Sprawdzimy,
     jak przekłada się to na dokładność omawianych metod.
    
     \begin{enumerate}
      \item W testach dla punktów, w których funkcja \texttt{exp(x)} przyjmuje małe wartości
      (1.2, $\pi$) wraz ze wzrostem liczby węzłów metoda 2. zwracała wyniki obarczone
      mniejszym błędem względnym.
      \item W obu wyżej wymienionych punktach dla 6 węzłów metoda 1. dawała zdecydowanie lepsze
      wyniki, dla 11 oraz 101 węzłów - metoda 2.
       \item Ciekawa obserwacja - w obu wyżej wymienionych punktach metoda 1. była
      najmniej dokładna nie dla 6, a 11 węzłów. Co więcej, w punkcie 1.2 najdokładniejsza była
      dla 6 węzłów.
      \item W punkcie 30 dla 100 węzłów równoodległych w przedziale [29, 31]
      metoda 2 zwraca zupełnie niedokładny wynik rzędu $10^{197}$, podczas gdy wartość dokładna
      pochodnej w tym punkcie jest rzędu $10^{13}$. Metoda 1 sprawdza się w tym punkcie dość dobrze
      (aczkolwiek powinniśmy zdefiniować, jakiej precyzji żądamy) - błąd względny jest rzędu 
      $10^{-5}$.
     \end{enumerate}
     
    \subsection{\texttt{2(x + 1)(x - 3)}}
     Wydaje się, że interpolacja wielomianu drugiego stopnia wielomianami 6, 11 czy 101 stopnia
     powinna generować na tyle małe błędy przy obliczaniu pochodnej, że będą one mniejsze niż precyzja
     arytmetyki BigFloat. Okazuje się, że tak nie jest.
     \begin{enumerate}
      \item W punkcie 1 (w którym pochodna powyższego wielomianu się zeruje) metoda 2 dla 101 węzłów
      zwraca błąd względny rzędu $10^{-61}$ (jest on bardzo mały, ale niezerowy). Inaczej ma się w przypadku
      11 i 101 węzłów oraz dla metody 1 i 6 węzłów.
      \item Jest to jedyna z testowanych funkcji, dla której obie metody już dla 6 węzłów
      obliczyły dokładną wartość pochodnej (w arytmetyce BigFloat).
      \item W punktach 5.123 i 9.9 metoda 2. okazała się niezawodna dla wszystkich zbiorów węzłów.
      
     \end{enumerate}

    \subsection{\texttt{ln(x)}}
    Czy asymptota w przedziale, w którym przybliżamy pochodną wpływa na dokładność przybliżeń?
    Sprawdźmy.
     \begin{enumerate}
      \item Zauważmy, że testowanie dla zbiorów węzłów zawierających zero nie ma sensu dla metody 2,
      ponieważ przy wyliczaniu ilorazów różnicowych bierzemy $f(0) = \infty$. Potwierdza to nasz pierwszy
      test dla wartości 0.1.
      \item Testowanie dla zbioru węzłów takiego, że punkt, w którym przybliżamy pochodną leży
      w przedziale $<x_0, x_1>$ nie ma sensu dla metody 1. Również potwierdza to nasz pierwszy test.
      \item Metoda 2. w punkcie 0.1 nawet jeśli zwraca wynik, to zupełnie niepoprawny
      \item Metoda 1. w punkcie 0.1 dla węzłów bez 0 zwraca relatywnie dobry wynik dla 1000 węzłów.
      \item W punkcie 1 metoda 2 również nie zwraca satysfakcjonujących wyników, jednak błędy
      są dużo mniejsze niż w punkcie 0.1.
     \end{enumerate}


   \section{Wnioski}
   Jak mogliśmy się przekonać w powyższych testach, nie dla każdej funkcji opisane przeze mnie metody
   dawały satysfakcjonujące rezultaty (\texttt{ln(x)}). Przy obliczaniu przybliżonej wartości pochodnej
   w punkcie z danego przedziału pomocne mogą okazać się informacje na temat tej funkcji:
   
   \begin{itemize}
    \item Czy funkcja w tym przedziale nie ma asymptoty/asymptot?
    \item Czy funkcja w tym przedziale jest określona w każym punkcie tego przedziału?
   \end{itemize}
   
   Dla funkcji, które spełniają oba powyższe kryteria możemy wyciągnąć następujące wnioski:
   \begin{itemize}
    \item Nie zawsze zwiększenie liczby węzłów powoduje zwiększenie precyzji przybliżenia
    \item W zdecydowanej większości metoda 2 dawała lepsze rezultaty niż metoda 1. Ważny wyjątek:
    metoda 1. zwracała lepszy wynik na przedziale, na którym funkcja bardzo szybko rośnie
    (\texttt{exp(x)}). Dlaczego w tym przypadku metoda 1. zwróciła relatywnie dobry wynik? Otóż
    gdy funkcja na pewnym przedziale bardzo szybko rośnie, bardziej przypomina na pewnym małym
    przedziale między węzłami funkcję liniową niż w przedziale, w którym funkcja zachowuje się inaczej
    (np. funkcja \texttt{exp(x)} w przedziale [0, 2] nie przypomina funkcji liniowej - jest 
    \textquotedblleft zaokrąglona\textquotedblright. A metoda 1. bazuje właśnie na założeniu,
    że przybliżana funkcja jest na pewnych małych przedziałach liniowa.
    \item Metoda 2 świetnie sprawdzała się dla wielomianu \texttt{2(x + 1)(x - 3)}. Ważne - wielomian był
    niższego stopnia niż wielomiany interpolacyjne Newtona z metody 2.
    
   \end{itemize}


   

\end{document}
