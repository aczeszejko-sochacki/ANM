\documentclass{article}

\title{Algorytm Molera-Morissona}
\author{Aleksander Czeszejko-Sochacki} 
\usepackage{amsmath}
\usepackage{polski}
\usepackage[utf8]{inputenc}
\usepackage{algorithm}
\usepackage{algorithmic}
\usepackage{hyperref}
\usepackage{setspace}
\usepackage{amssymb}

\setlength\parindent{24pt}

\begin{document}
    \maketitle
    \pagenumbering{arabic}
    \section{Wprowadzenie}
        \indent Algorytm Molera-Morissona to iteracyjna metoda obliczania 
	wyrażeń postaci $\sqrt{a^2 + b^2}$\footnote{a, b $\in\mathbb{R}$, najlepiej gdy a, b $\in\mathbb{Q}$, ponieważ przybliżona reprezentacja
	liczb niewymiernych może powiększać ewentualny błąd}. Tego typu funkcje 
        pojawiają się nadzwyczaj często, między innymi przy obliczaniu:
        \begin{itemize}
            \item przeciwprostokątnej
            \item modułu liczby zespolonej
            \item normy wektora
	    \item zmiany układu współrzędnych z eukidesowych na biegunowe
        \end{itemize}
	\par

        \subsection{Pseudokod}
            \begin{algorithm}
            \caption{Algorytm iteracyjny Molera-Morissona}
                \begin{algorithmic}
                    \STATE $p := max(|a|, |b|)$\\
                           $q := min(|a|, |b|)$
                    \WHILE{wartość q jest znacząca}
                        \STATE $r:= (\frac{q}{p})^2$\\
                               $s := \frac{r}{4 + r}$\\
                               $p := p + 2*s*p$\\
                               $q := s * q$\\
                    \ENDWHILE
                    \RETURN p

                \end{algorithmic}
            
            \end{algorithm}
	    
	    \indent Od razu możemy zauważyć zalety algorytmu - nie wymaga on używania skomplikowanych operacji arytmetycznych,
	    a każda iteracja składa się z kilku elementarnych działań i przypisów. 
        \newpage
	
        \subsection{Liczba iteracji}
            \indent Dowodzi się, że po 3 iteracjach algorytmu błąd względny jest na poziomie $0.5 * 10^{-20}$, 
            co uznamy za wystarczającą dokładność w naszym artykule. W przypadku chęci szerszej analizy zbieżności i numerycznej poprawności
	    odsyłamy Czytelnika do rozdziału czwartego \url{https://blogs.mathworks.com/images/cleve/moler_morrison.pdf}
            \par
	    
	\subsection{Opis użytych metod}
	    \indent W swoim sprawozdaniu zaprezentuję obliczenia wykonane w języku Julia.
	    Przeprowadzone zostały w dwóch arytmetykach:
	    \begin{enumerate}
	        \item double, tzn. Float64
		\item BigFloat(128)
	    \end{enumerate}
            \par
	    
	    \indent Dlaczego takie arytmetyki? Precyzja arytmetyki Float64 może okazać się niewystarczająca, ponieważ wynosi ona
	    \[\frac{1}{2}2^{-52} = \frac{1}{8}{(2^{-10})}^5 \approx 10^{-16}\]
	    
	    \begin{center}
	       \begin{tabular}{| l | l | l |}
	           \hline
		   Numer iteracji  &  Błąd względny dla pary (3, 4)  &  Błąd względny dla pary (-5, 12)\\
		   \hline
		   2.              &  5.162349481224737e-9,         &  5.241619104568739e-13\\
		   3.              &  1.7763568394002506e-16,       &  1.3664283380001927e-16\\
		   \hline
	       \end{tabular}
	    \end{center}
	    
	    Powyższe wyniki zgadzają się z naszym wnioskiem. Powyższe błędy po trzeciej iteracji powinny być rzędu najwyżej $10^{-21}$.
	    Powinniśmy dobrać taką arytmetykę, której precyzja będzie mniejsza niż $\frac{1}{2}2^{-20}$. Zakładając, że zwiększając w Julii liczbę bitów przydzielanych dla arytmetyki dwukrotnie, w przybliżeniu
	    dwukrotnie zwiększa się liczba bitów przydzielanych na mantysę, precyzja arytmetyki BigFloat(128) wynosi około:
	    \[\frac{1}{2}2^{-100} = \frac{1}{2}{(2^{-10})}^{10} \approx 10^{-31}\] co zdecydowanie wystarczy na wychwycenie ewentualnego
	    błędu algorytmu.
	    
	    
	    Za wartość dokładną $\sqrt{a}$ będę uznawał wartość funkcji sqrt(a) w arytmetyce BigFloat(128)
	    \par
	   
    
    \section{Analiza Algorytmu}	   
	   \subsection{Testownie algorytmu dla wybranych danych}
	   \paragraph{Poniżej przeprowadzimy testy dla następująch danych:}
	   
	   \[(3, 4), (-5, 12), (7, -24), (1, 1), (1000000000,2), (71075075103, 1000000000)\] 
	   
	   
	   \paragraph{Testy (trzy iteracje:)}
	   
	   \subparagraph{(3, 4), dokładna wartość: 5.000000000000000000000000000000000000000\\ Małe wartości obu danych, całkowity wynik}
	   \begin{center}
	       \begin{tabular}{| l | l | l |}
	           \hline
		       Numer iteracji  &  Wynik algorytmu (Float64)  &  Wynik algorytmu (BigFloat(128))\\
		   \hline
		       1               &  4.986301369863014          &  4.986301369863013698630136986301369863002\\
		       2               &  4.999999974188253          &  4.999999974188252149492661061886530575595\\
		       3               &  5.000000000000001          &  4.999999999999999999999999828030176654495\\
		   \hline
	       \end{tabular}
	   \end{center}
	   \vspace{1cm}
	   
	   \subparagraph{(-5, 12), dokładna wartość: 1.300000000000000000000000000000000000000e+01\\ Pierwsza dana ujemna, całkowity wynik}
	   \begin{center}
	       \begin{tabular}{| l | l | l |}
	           \hline
		       Numer iteracji  &  Wynik algorytmu (Float64)  &  Wynik algorytmu (BigFloat(128))\\
		   \hline
		       1               &  12.998336106489186         &  1.299833610648918469217970049916805324457e+01\\
		       2               &  12.999999999993186         &  1.29999999999931842560000017867063951355e+01\\
		       3               &  13.000000000000002         &  1.299999999999999999999999999999999999948e+01\\
		   \hline
	       \end{tabular}
	   \end{center}
	   \vspace{1cm}
	   
	   \subparagraph{(7, -24), dokładna wartość: 2.500000000000000000000000000000000000000e+01\\ Druga dana ujemna, całkowity wynik}
	   \begin{center}
	       \begin{tabular}{| l | l | l |}
	           \hline
		       Numer iteracji  &  Wynik algorytmu (Float64)  &  Wynik algorytmu (BigFloat(128))\\
		   \hline
		       1               &  24.999575010624735         &  2.499957501062473438164045898852528686783e+01\\
		       2               &  24.99999999999997          &  2.499999999999996929526990921773880960337e+01\\
		       3               &  25.000000000000004         &  2.500000000000000000000000000000000000000e+01\\
		   \hline
	       \end{tabular}
	   \end{center}
	   \vspace{1cm}
	   
	   \subparagraph{(1, 1), dokładna wartość: 1.414213562373095048801688724209698078569\\ Równe obie dane, nie dają całkowitego wyniku}
	   \begin{center}
	       \begin{tabular}{| l | l | l |}
	           \hline
		       Numer iteracji  &  Wynik algorytmu (Float64)  &  Wynik algorytmu (BigFloat(128))\\
		   \hline
		       1               &  1.4                        &  1.399999999999999999999999999999999999999\\
		       2               &  1.4142131979695431         &  1.414213197969543147208121827411167512687\\
		       3               &  1.4142135623730951         &  1.41421356237309504879564008075425994635\\
		   \hline
	       \end{tabular}
	   \end{center}
	   \vspace{1cm}
	   
	   \subparagraph{(1000000000, 2), dokładna wartość: 1.000000000000000001999999999999999998001e+09\\ Duża pierwsza dana, duża względna różnica, nie dają
	   całkowitego wyniku}
	   \begin{center}
	       \begin{tabular}{| l | l | l |}
	           \hline
		       Numer iteracji  &  Wynik algorytmu (Float64)  &  Wynik algorytmu (BigFloat(128))\\
		   \hline
		       1               &  1.0e9                      &  1.000000000000000001999999999999999998001e+09\\
		       2               &  1.0e9                      &  1.000000000000000001999999999999999998001e+09\\
		       3               &  1.0e9                      &  1.000000000000000001999999999999999998001e+09\\
		   \hline
	       \end{tabular}
	   \end{center}
	   \vspace{1cm}
	   
	   \subparagraph{(71075075103, 1000000000), dokładna wartość: \\7.108210956982840179871838428090505048092e+10\\ Duże obie dane, nie dają całkowitego wyniku}
	   \begin{center}
	       \begin{tabular}{| l | l | l |}
	           \hline
		       Numer iteracji  &  Wynik algorytmu (Float64)  &  Wynik algorytmu (BigFloat(128))\\
		   \hline
		       1               &  7.108210956981117e10       &  7.108210956981117607726957458289901508523e+10\\
		       2               &  7.10821095698284e10        &  7.108210956982840179871838428090505048092e+10\\
		       3               &  7.10821095698284e10        &  7.108210956982840179871838428090505048132e+10\\
		   \hline
	       \end{tabular}
	   \end{center}
           \vspace{1cm}
	   
	   \indent Możemy w powyższych przykładach zaobserwować następujące zjawiska:
	   \begin{itemize}
	       \item Zmniejszenie precyzji arytmetyki powoduje zwiększenie liczby cyfr znaczących
	       \item Algorytm nie wyliczył oczekiwanych wartości całkowitych dla pierwszych trzech par
	     \end{itemize}
	   
	   Więcej obserwacji będziemy w stanie wyciągnąć po analizie błędów względnych.
	   \par
	   
	   \paragraph{Błędy względne}\footnote{dla przejrzystości zmniejszyłem liczbę nieznaczących zer w błędach w arytmetyce BigFloat(128); $\tilde{y}$ - wynik algorytmu, 
	    y - wartość obliczona przy pomocy funkcji sqrt()}
	   
	   \begin{center}
	       \begin{tabular}{| l | l | l |}
	           \hline
		   Dane                       &  $\frac{|\Delta y|}{y}$, $\tilde{y}$ we Float64           &  $\frac{|\Delta y|}{y}$, $\tilde{y}$ w BigFloat(128)\\
		   \hline
		       (3, 4)                     &  1.776356839400250464677810668945312500001e-16  &  0.0\\
		       (-5, 12)                   &  1.366428338000192665136777437650240384618e-16  &  0.0\\
		       (7, -24)                   &  1.421085471520200371742248535156250000001e-16  &  0.0\\
		       (1, 1)                     &  0.000000000000000000000000000000000000000      &  0.0\\
		       (1000000000, 2)            &  0.000000000000000000000000000000000000000      &  0.0\\
		       (71075075103, 1000000000)  &  0.000000000000000000000000000000000000000      &  0.0\\
		   \hline
	       \end{tabular}
	   \end{center}
	   \vspace{1cm}
	   
	   \indent Obserwacje (w powyższych przykładach):
	   \begin{itemize}
	     \item Błędy w arytmetyce Float64 są najwyżej rzędu $10^{-16}$, co zgadza się z przybliżoną wartością
	     precyzji tej arytmetyki obliczoną we wprowadzeniu
	     \item W niektórych przypadkach błędy są zerowe (lub mniejsze niż precyzja arytmetyki)
	     \item W arytemtyce BigFloat(128) wszyskie błędy są zerowe (lub mniejsze niż precyzja arytmetyki)
	     \item Błedy w arytmetyce BigFloat(128) są niewiększe niż we Float64
	   \end{itemize}
	   
	   \newtheorem{hip}{Hipoteza}
	   \begin{hip}
	      Dla dostatecznie dużej precyzji algorytm zwraca \underline{matematycznie dokładny} wynik dla dowolnych danych
	      (nie wiadomo, czy taka precyzja jest osiągalna na każdej maszynie).
	   \end{hip}

	   
	\subsection{Ciekawostka}
	    \indent Algorytm zachowuje się ciekawie, gdy opuścimy wartości bezwzględne w inicjalizacji zmiennych. Po trzeciej iteracji
	    \[\sqrt{(-7)^2 + 24^2} \neq \sqrt{7^2 + (-24)^2}\]
	    natomiat po czwartej
	    \[\sqrt{(-7)^2 + 24^2} = \sqrt{7^2 + (-24)^2}\]
	    Zachęcam Czytelnika do przeprowadzenia własnych eksperymentów.
	    \par
        \subsection{Zalety stosowania algorytmu}
	    \indent Algorytm iteracyjny Molera-Morissona ma bardzo przyzwoity wykładnik zbieżności - z każdą iteracją liczba cyfr znaczących
	    zwiększa się około trzykrotnie. Niewątpliwą jego przewagą nad liczeniem wyrażenia przy pomocy funkcji sqrt() jest fakt, że w algorytmie
	    nie podnosimy danych do kwadratu, przez co ma dużo mniejszą złożoność pamięciową. Złożoność czasowa zależy od czasu wykonywania elementarnych
	    operacji (chyba że chcemy wykonać więcej niż trzy iteracje).
	    
	    
    \section{Zasosowanie algorytmu Molera-Morissona do obliczania normy wektora}
        \indent Normą wektora x nazywamy wyrażenie:
	\[\Arrowvert x \Arrowvert = \sqrt{\sum_{i=1}^{n}x_i^2}\]
	Możemy ją jednak zapisać następująco:
	
	\begin{equation}
	  \begin{split}
	    \sqrt{\sum_{i=1}^{n}x_i^2} &= \underbrace{\sqrt{x_1^2 + x_2^2 + \cdots + x_n^2}}_{n}\\ &= \sqrt{\sqrt{(\sqrt{x_1^2 + x_2^2})^2 + x_3^2}^2 +
	      \cdots + x_n^2}\\ &= \sqrt{\sqrt{\sqrt{\sqrt{(\sqrt{x_1^2 + x_2^2})^2 + x_3^2}^2 +
		  \cdots + x_{n-1}}^2 + x_n^2}} 
	  \end{split}
	\end{equation}
        
	\subsection{Pseudokod obliczania normy wektora}
	
	   \indent Powyższe rozpisanie umożliwia nam zapisanie algorytmu:
	   \begin{algorithm}
            \caption{Algorytm iteracyjny Molera-Morissona}
                \begin{algorithmic}
		  \STATE result = $x_1$
		  \FOR{i in 2, 3, $\cdots$, n}
		      \STATE result = Moler-Morrison(result, $x_i$)
		  \ENDFOR
		  \RETURN result
                \end{algorithmic}        
            \end{algorithm}
	    
	    Dokładność otrzymanego wyniku będzie zależna od dokładności samego algorytmu, jak i kolejności sumowania, 
	    jednak problem kolejności sumowania dla uzyskania najmniejszego błędu to temat obszerny i nie będzie on rozważany
	    w tym artykule.
            \par
	   
	    \begin{thebibliography}{99}
	      \bibitem{pa} C.~Moler, D.~Morrison: Replacing Square Roots by Pythagorean Sums
	    \end{thebibliography}

\end{document}\documentclass[10pt,a4paper]{article}